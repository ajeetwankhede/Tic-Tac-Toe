\href{https://travis-ci.com/ajeetwankhede/Tic-Tac-Toe}\texttt{ } 



\subsection*{Project Overview}

The task is to build an AI agent to learn to play Tic Tac Toe game and to implement object oriented programming concepts.

\subsection*{Breif about Tic Tac Toe (\href{https://en.wikipedia.org/wiki/Tic-tac-toe}\texttt{ wikipedia})}

Tic-\/tac-\/toe (American English), noughts and crosses (British English), or Xs and Os is a paper-\/and-\/pencil game for two players, X and O, who take turns marking the spaces in a 3×3 grid. The player who succeeds in placing three of their marks in a horizontal, vertical, or diagonal row first wins the game.

 

\subsection*{Breif about AI agent}

In this project Q learning is implemented to train an AI agent. A general introduction can be obtained from \href{https://www.cs.swarthmore.edu/~meeden/cs63/f11/ml-ch13.pdf}\texttt{ Chapter 13} of Machine Learning, Tom Mitchell. The implementation is based on a pseudo code given by Meeden \href{https://www.cs.swarthmore.edu/~meeden/cs63/f11/lab6.php}\texttt{ C\+S63 Lab 6}. After training the agent for 200,000 self-\/play games ($\sim$1 min), it reaches an optimal policy. The agent learns to choose the center move if vacant leading to either a win or draw game, as seen in \href{https://www.cs.dartmouth.edu/~lorenzo/teaching/cs134/Archive/Spring2009/final/PengTao/final_report.pdf}\texttt{ Reinforcement Learning in Tic-\/\+Tac-\/\+Toe Game and Its Similar Variations}. When tested against a random player (randomly chooses valid moves) for 1000 games, AI agent either wins ($\sim$99\%) or draws ($\sim$1\%) the game but never loses.

\subsection*{Software design}


\begin{DoxyEnumerate}
\item U\+ML diagram\+:
\end{DoxyEnumerate}

 


\begin{DoxyEnumerate}
\item Activity diagram\+:
\end{DoxyEnumerate}

 

\subsection*{Link for S\+IP document}

\href{https://docs.google.com/spreadsheets/d/1osaNjtBZ5rFgGtWi5gnai63RkGZE8U25gTj9zXKbLFg/edit#gid=0}\texttt{ S\+IP Enactment Sheet}

\subsection*{Dependencies}

The game has dependency on following packages\+:
\begin{DoxyEnumerate}
\item \href{https://cmake.org/}\texttt{ cmake}
\item \href{https://github.com/google/googletest}\texttt{ googletest}
\end{DoxyEnumerate}

\#\# Standard install via command-\/line 
\begin{DoxyCode}{0}
\DoxyCodeLine{git clone --recursive https://github.com/ajeetwankhede/Tic-Tac-Toe.git}
\DoxyCodeLine{cd <path to repository>}
\DoxyCodeLine{mkdir build}
\DoxyCodeLine{cd build}
\DoxyCodeLine{cmake ..}
\DoxyCodeLine{make}
\DoxyCodeLine{Run tests: ./test/cpp-test}
\DoxyCodeLine{Run program: ./app/shell-app}
\end{DoxyCode}


\#\# Building for code coverage 
\begin{DoxyCode}{0}
\DoxyCodeLine{sudo apt-get install lcov}
\DoxyCodeLine{cmake -D COVERAGE=ON -D CMAKE\_BUILD\_TYPE=Debug ../}
\DoxyCodeLine{make}
\DoxyCodeLine{make code\_coverage}
\end{DoxyCode}
 This generates a index.\+html page in the build/coverage sub-\/directory that can be viewed locally in a web browser. Below is a screenshot of code coverage report\+: 

 

\subsection*{How to generate Doxygen report}


\begin{DoxyCode}{0}
\DoxyCodeLine{sudo apt-get install doxygen}
\DoxyCodeLine{cd <path to repository>}
\DoxyCodeLine{mkdir Doxygen}
\DoxyCodeLine{cd Doxygen}
\DoxyCodeLine{doxygen -g <config\_file\_name>}
\DoxyCodeLine{gedit <config\_file\_name>}
\end{DoxyCode}
 Update P\+R\+O\+J\+E\+C\+T\+\_\+\+N\+A\+ME, P\+R\+O\+J\+E\+C\+T\+\_\+\+B\+R\+I\+EF, and I\+N\+P\+UT fields in the configuration file. Then run the following command to generate the documentations. 
\begin{DoxyCode}{0}
\DoxyCodeLine{doxygen <config\_file\_name>}
\end{DoxyCode}
 In doxygen folder, config file (tictactoe) and genertaed reports are saved as html and latex format.

\subsection*{Tools for static code analysis}

Cpplint is used for static code analysis to identify potential source code issues that conflict with the Google C++ style guide. Further, cppcheck is used to detect various kinds of bugs in the code. The results generated by these tools are kept in results directory. 